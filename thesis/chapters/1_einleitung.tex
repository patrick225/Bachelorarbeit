\chapter{Einleitung}
\label{ch:einleitung}

Die Geschichte der Roboter reicht bis in die Antike zurück. Schon dort gab es erste Versuche mit Automaten, die zum Beispiel Musik spielen sollten oder automatisch Theater spielen konnten. Mit dem Verlust der antiken Kulturen gingen jedoch auch die Kenntnisse über die Automation verloren. Erst im 13. Jahrhundert wurde ein Buch eines arabischen Ingenieurs bis in die westliche Welt bekannt, was Gerüchten zufolge auch Leonardo da Vinci für die Automation inspiriert haben soll. 
Roboter wie wir sie kennen gibt es erst seit Mitte des 20. Jahrhunderts. Der technische Wendepunkt, der mit der Erfindung des Transistors kam, machte es erst möglich, elektrische Schaltungen in einer Größe zu fertigen, die für Roboter notwendig ist. Seit diesem Zeitpunkt liegt ein wertvoller und nicht mehr wegzudenkender Industriezweig auf dem Gebiet der Robotik. Mit dem Einstieg der Robotik in die Industrie wurde die Produktion um ein vielfaches schneller und präziser, als sie von Menschenhand je vorgenommen werden könnte. Auch in Bereichen oder Umgebungen, die für Menschen lebensgefährlich oder ohnehin lebensunmöglich sind, spielen Roboter eine bedeutende Rolle. Ein Beispiel hierfür ist der Mars-Rover. Der Mars-Rover ist ferngesteuertes Fahrzeug, welcher für die Marsforschung verwendet wird und größten Teils von der Erde aus gesteuert wird. Wie die meisten ferngesteuerten Fahrzeuge ist er mit einer Vielzahl von Sensoren und Werkzeugen ausgestattet. 

Diese Bachelorarbeit beschäftigt sich mit dem Thema der Steuerung von mobilen Robotern mit dem Ziel, ein Roboter-Fussballspiel zu ermöglichen. Auf einer Tischplatte in der Universität sollen zwei Roboter platziert werden. An den beiden Enden der Tischplatte werden Tore montiert, ähnlich wie man es beim Tischfussball kennt. Mithilfe von mobilen Endgeräten sollen nun die Roboter angesteuert werden können und ein Zwei-Spieler Fussballspiel ermöglicht werden. Die Roboter verfügen über einen Schussapparat, der es möglich macht den Ball zu beschleunigen. Sobald der Ball in ein Tor befördert wurde, wird ein Tor automatisch erkannt und auf den mobilen Endgeräten, welche die Steuercontroller bilden, angezeigt. Diese Steuercontroller können entweder ein Android-Gerät oder jedes beliebige andere Gerät sein, das über eine Tastatur und einen Webbrowser verfügt. Über diese Steuercontroller ist es möglich die Roboter zu navigieren und einen Schuss zu tätigen. Hierfür werden verschiedene Steuerarten bereitgestellt. 
Die Roboter bewegen sich vollkommen kabellos, was die Verwendung eines Akkus notwendig macht. Sobald der Akku einen Mindestprozentsatz unterschreitet, wird automatisch die Ladestation angefahren, damit sich der Akku des Roboters selbstständig, also ohne menschliches Eingreifen, lädt.
Die Fertigung und Implementierung der Roboter ist Teil einer anderen Bachelorarbeit, auf die hier nur Oberflächlich eingegangen wird.               

% hier kommt noch ein kurzer überblick über die arbeit, welche kapitel folgen und was in welchen kapiteln ungefähr beschrieben wird
                                                                                                                                                                                                                                                                                                                                                                                                                                                                                                                                                                                                                                                                                                                                                                                                                                                                                                                                                                                                                                                                                                                                                                                                                                                                                                                                                                                                                                                                                                                                                                                                                                                                                                                                                                                                                                                                                                                                                                                                                                                                                                                                                                                                                                                                                                                                                                                                                                                                                                                                                                                                                                                                                                                                                                                                                                                                                                                                                                                                                                                                                                                                                                                                                                                                                                                                                                                                                                                                                                                                                                                                                                                                                                                                                                                                                                                                                                                                                                                                                                                                                                                                                                                                                                                                                                                                                                                                                                                                                                                                                                                                                                                                                                                                                                                                                                                                                                     