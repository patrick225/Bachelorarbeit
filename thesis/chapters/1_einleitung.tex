\chapter{Einleitung}
\label{ch:einleitung}

Die Geschichte der Roboter reicht bis in die Antike zurück. Schon dort gab es erste Versuche mit Automaten, die zum Beispiel Musik spielen sollten oder automatisch Theater spielen konnten. Mit dem Verlust der antiken Kulturen gingen jedoch auch die Kenntnisse über die Automation verloren. Erst im 13. Jahrhundert wurde ein Buch eines arabischen Ingenieurs bis in die westliche Welt bekannt, was Gerüchten zufolge auch Leonardo da Vinci für die Automation inspiriert haben soll. 
Roboter wie wir sie kennen gibt es erst seit Mitte des 20. Jahrhunderts. Der technische Wendepunkt, der mit der Erfindung des Transistors kam, machte es erst möglich, elektrische Schaltungen in einer Größe zu fertigen, die für Roboter notwendig ist. Seit diesem Zeitpunkt liegt ein wertvoller und nicht mehr wegzudenkender Industriezweig auf dem Gebiet der Robotik. Mit dem Einstieg der Robotik in die Industrie wurde die Produktion um ein vielfaches schneller und präziser, als sie von Menschenhand je vorgenommen werden könnte. Auch in Bereichen oder Umgebungen, die für Menschen lebensgefährlich oder ohnehin lebensunmöglich sind, spielen Roboter eine bedeutende Rolle. Ein Beispiel hierfür ist der Mars-Rover. Der Mars-Rover ist ferngesteuertes Fahrzeug, welcher für die Marsforschung verwendet wird und größten Teils von der Erde aus gesteuert wird. Wie die meisten ferngesteuerten Fahrzeuge ist er mit einer Vielzahl von Sensoren und Werkzeugen ausgestattet. 

Seit Ende der 90-er Jahre gibt es einen weltweiten Wettbewerb namens RoboCup. Bei RoboCup geht es darum, ein Fussballspiel von Robotern austragen zu lassen. Diese Bachelorarbeit beschäftigt sich ebenfalls mit dem Thema der Steuerung von mobilen Robotern mit dem Ziel, ein Roboter-Fussballspiel zu ermöglichen. Im Gegensatz zum RoboCup, bei dem die Roboter entweder autonom, oder von einem zentralen Computer gesteuert werden, der mittels einer Kamera über dem Spielfeld die Bewegungen und Situationen erfasst, wird in dieser Bachelorarbeit die Steuerung der Roboter vom Menschen übernommen. Auf einer Tischplatte in der Universität sollen zwei Roboter platziert werden. An den beiden Enden der Tischplatte werden Tore montiert, ähnlich wie man es beim Tischfussball kennt. Mithilfe von mobilen Endgeräten sollen nun die Roboter angesteuert werden können und ein Zwei-Spieler Fussballspiel ermöglicht werden. Die Roboter verfügen über einen Schussapparat, der es möglich macht den Ball zu beschleunigen. Sobald der Ball in ein Tor befördert wurde, wird ein Tor automatisch erkannt und auf den mobilen Endgeräten, welche die Steuercontroller bilden, angezeigt. Diese Steuercontroller können entweder ein Android-Gerät oder jedes beliebige andere Gerät sein, das über eine Tastatur und einen Webbrowser verfügt. Über diese Steuercontroller ist es möglich die Roboter zu navigieren und einen Schuss zu tätigen. Hierfür werden verschiedene Steuerarten bereitgestellt. 
Die Roboter bewegen sich vollkommen kabellos, was die Verwendung eines Akkus notwendig macht. Sobald der Akku einen Mindestprozentsatz unterschreitet, wird automatisch die Ladestation angefahren, damit sich der Akku des Roboters selbstständig, also ohne menschliches Eingreifen, lädt.
Die Fertigung und Implementierung der Roboter ist Teil einer anderen Bachelorarbeit, auf die hier nur Oberflächlich eingegangen wird.               

% hier kommt noch ein kurzer überblick über die arbeit, welche kapitel folgen und was in welchen kapiteln ungefähr beschrieben wird




\section{Anforderungen}
Das Ziel der Bachelorarbeit ist es, ein Fussballspiel für zwei Spieler zu realisieren. Hierfür werden zwei Roboter auf einem begrenzten Spielfeld positioniert. Jeweils am Rand des Spielfelds wird ein Tor montiert, in das der Ball befördert werden soll. Mittels mobilen Endgeräten und herkömmlichen PCs soll es möglich sein, die Steuerung über einen der Roboter zu übernehmen. Sobald beide Roboter mit einem Steuergerät verbunden sind, soll ein Spiel gestartet werden und die Torerkennung funktionieren. Das bedeutet, sobald ein Spieler ein Tor erzielt hat, erkennt das System dieses und erhöht den Spielstand dementsprechend. Mittels einer Kamera, die auf den Robotern montiert ist, soll eine Videoübertragung aus Sicht der Roboter stattfinden. Dadurch soll es auch möglich sein, die Roboter zu steuern, ohne sich in Sichtkontakt mit den Robotern zu befinden. Da die Roboter über einen Akku verfügen, soll die Kommunikation kabellos erfolgen, damit ein freies Fahren der Roboter sichergestellt ist. Weil der Akku eine begrenzte Kapazität hat, soll der Roboter bei niedrigem Akkustand selbstständig die Ladevorrichtung anfahren und während der Ladezeit keine Steuerbefehle annehmen. Aufgrund der Interaktion mit menschlichen Benutzern, ist es wichtig, dass der Roboter eine möglichst rechtzeitige Steuerung darstellt, um Verzögerungen und die dadurch entstehende Unkontrollierbarkeit zu vermeiden.

\section{Schnittstelle zum Roboter}
\label{sec:schnittstelle}
Die Schnittstelle zum Roboter definiert auch die Abgrenzung dieser Bachelorarbeit zu der von Alexander Ulbrich. Durch den Pool der möglichen Befehle an den Roboter wird klar, was der Roboter bereits kann und was durch die Steuereinheit erreicht werden muss. 

Da der Roboter eine Byte-Orientierte Kommunikationsweise erfordert mussten Protokolle für beide Übertragungsrichtungen definiert werden.
Im Folgenden sind die Befehle aufgelistet, die der Roboter vom Server empfängt. \\

\begin{figure}[!h]
\begin{tabular}{||p{0.15\textwidth}||p{0.22\textwidth}||p{0.525\textwidth}||}
	\hline Bytenummer & Befehl & Beschreibung \\ 
	\hline 0 & Start Byte &  Jeder Befehl beginnt mit 0xFF\\ 
	\hline 1 & Kamera & 0x01 aktiviert die Kamera, 0x00 deaktiviert sie \\ 
	\hline 2 & Schussapparat & 0x01 löst einen Schuss aus, zu schnelle Wiederholungen werden vom Roboter ignoriert \\ 
	\hline 3 & Motorleistung links & Setzt die Leistung des linken Antriebs auf diesen Wert (zwischen 0 und 100 \% in Hex, also 0x00 - 0x64) \\ 
	\hline 4 & Motorleistung rechts & Setzt die Leistung des rechten Antriebs auf diesen Wert (zwischen 0 und 100 \% in Hex, also 0x00 - 0x64) \\ 
	\hline 5 & Checksumme & Dieses Feld ermöglicht die Erkennung von fehlerhaften Paketen. Sowohl dem Server, als auch dem Roboter ist die Berechnung hierfür bekannt. \\
	\hline
\end{tabular}  
\caption{Befehle, die der Roboter annimmt}
\label{tab:serv_to_robot}
\end{figure}


Hier wird ersichtlich, dass selbst für geradeaus fahren keine Funktion oder derartiges bereit steht, lediglich die beiden Motoren können unabhängig voneinander angesteuert werden. Dadurch muss jegliche Logik, die das Steuern des Roboters ermöglicht, in dieser Arbeit behandelt werden. \\

In der anderen Richtung, also Befehle die vom Roboter aus gehen, gibt es lediglich einige Statusnachrichten und die Bilddaten. Auch diese sind in Tabelle \ref{tab:robot_to_serv} aufgeführt.



\begin{figure}[!h]
	\begin{tabular}{||p{0.15\textwidth}||p{0.22\textwidth}||p{0.525\textwidth}||}
		\hline Bytenummer & Befehl & Beschreibung \\ 
		\hline 0 & Start Byte &  Jeder Befehl beginnt mit 0xFF\\ 
		\hline 1 & Akkustand & Enthält den aktuellen Ladestand des Akkus (zwischen 0 und 100 \% in Hex, also 0x00 - 0x64) \\ 
		\hline 2 & Goal-Flag & Enthält Werte zwischen 0x00 und 0x03. Das entspricht den Situationen: Tor in Sicht, Richtungsweisung in Sicht (links und rechts) und kein Tor in Sicht \\ 
		\hline 3 - 9 & Nicht-Implementiert & An dieser Stelle waren ursprünglich mehrere Stati eingeplant, jedoch aus Zeit- und Aufwandgründen nicht weiter beachtet \\ 
		\hline 10 + 11 & Bildlänge & In diesen beiden Bytes wird die Bildlänge mitgeteilt, die das Paket ab diesem Byte mit sich bringt \\ 
		\hline 12 - ? & Bilddaten & Je nach Bedarf werden noch zusätzliche Bilddaten an das Paket angehängt \\
		\hline
	\end{tabular}  
	\caption{Befehle, die der Roboter versendet}
	\label{tab:robot_to_serv}
\end{figure}                         

Außerdem hat man sich auf eine Befehlsfrequenz von 10Hz geeinigt. Bei dieser Frequenz wird der Roboter nicht überfordert und der Benutzer hat dennoch mit keinen Verzögerungserscheinungen zu kämpfen. auch die Statusnachrichten kommen in diesem Intervall, es sei denn, es werden noch Bilddaten angehängt, dann sendet der Roboter ohne Unterbrechungen.                                                                                                                                                                                                                                                                                                                                                                                                                                                                                                                                                                                                                                                                                                                                                                                                                                                                                                                                                                                                                                                                                                                                                                                                                                                                                                                                                                                                                                                                                                                                                                                                                                                                                                                                                                                                                                                                                                                                                                                                                                                                                                                                                                                                                                                                                                                                                                                                                                                                                                                                                                                                                                                                                                                                                                                                                                                                                                                                                                                                                                                                                                                                                                                                                                                                                                             
                                                                                                                                                                                                                                                                                                                                                                                                                                                                                                                                                                                                                                                                                                                                                                                                                                                                                                                                                                                                                                                                                                                                                                                                                                                                                                                                                                                                                                                                                                                                                                                                                                                                                                                                                                                                                                                                                                                                                                                                                                                                                                                                                                                                                                                                                                                                                                                                                                                                                                                                                                                                                                                                                                                                                                                                                                                                                                                                                                                                                                                                                                                                                                                                                                                                                                                                                                                                                                                                                                                                                                                                                                                                                                                                                                                                                                                          
                                                                                                                                                                                                                                                                                                                                                                                                                                                                                                                                                                                                                                                                                                                                                                                                                                                                                                                                                                                                                                                                                                                                                                                                                                                                                                                                                                                                                                                                                                                                                                                                                                                                                                                                                                                                                                                                                                                                                                                                                                                                                                                                                                                                                                                                                                                                                                                                                                                                                                                                                                                                                                                                                                                                                                                                                                                                                                                                                                                                                                                                                                                                                                                                                                                                                                                                                                                                                                                                                                                                                                                                                                                                                                                                                                                                                                                                                                                                                                                                                                                                                                                                                                                                                                                                                                                                                                                                                                                                                                                                                                                                                                                                                                                                                                                                                                                                                                