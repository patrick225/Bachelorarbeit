\chapter{Einleitung}
\label{ch:einleitung}

Bereits seit einigen Jahren finden sich in der Unterhaltungsindustrie zahlreiche Angebote von ferngesteuerten Robotern. Hierbei kann unterschieden werden zwischen den Arten, bei denen der Mensch die Steuerung übernimmt, oder denen bei denen die Roboter autonom fahren und handeln. 1997 wurde ein Wettbewerb gegründet, bei dem es Ziel ist, eine Art selbstspielende Roboter-Fussballmannschaft zu erstellen. Die Rede ist von RoboCup \cite{ROBOCUP}. Hauptaugenmerk liegt hierbei natürlich in der Entwicklung der Roboter und der autonomen Steuerung. \\
Bei einer vom Menschen initiierten Steuerung, wie zum Beispiel bei ferngesteuerten Autos, stellt sich gleichzeitig die Frage der Übertragungstechnologie. Soll es möglich sein das Fahrzeug oder den Roboter auch ohne direkten Sichtkontakt zu steuern oder nicht? Was bedeutet Rechtzeitigkeit in diesem Kontext? \\
Ziel dieser Bachelorarbeit ist es, eine von Menschen initiierte Steuerung eines eigens entwickelten Roboters zu entwickeln, die sich mit diesen Fragen beschäftigt. Hierzu sind noch weitere Rahmenbedingungen zu definieren, um eine sinnvolle Entscheidungsfindung zu gewährleisten.


Auf einer Tischplatte in der Universität sollen zwei Roboter platziert werden. An den beiden Enden der Tischplatte werden Tore montiert, ähnlich wie man es beim Tischfussball kennt. Da seit einiger Zeit Smartphones und Laptops allgegenwärtig sind, sollen diese die Steuercontroller bilden. Mithilfe davon sollen nun die Roboter angesteuert werden können und ein Zwei-Spieler Fussballspiel ermöglicht werden. Die Roboter verfügen über einen Schussapparat, der es möglich macht den Ball zu beschleunigen. Sobald der Ball in ein Tor befördert wurde, wird ein Tor automatisch erkannt und auf den Steuercontrollern angezeigt. Diese Steuercontroller können entweder ein Android-Gerät oder jedes beliebige andere Gerät sein, das über eine Tastatur und einen Webbrowser verfügt. Über diese Steuercontroller ist es möglich die Roboter zu navigieren und einen Schuss zu tätigen. Hierfür werden verschiedene Steuerarten bereitgestellt. 
Die Roboter bewegen sich vollkommen kabellos, was die Verwendung eines Akkus notwendig macht. Sobald der Akkustand unter einen bestimmten Schwellenwert fällt, wird der Roboter von der Steuerungslogik automatisch an die Ladestation geführt, damit sich der Akku des Roboters lädt, ohne den Roboter manuell dort hin bewegen zu müssen. Über die Kamera, die an den Robotern befestigt ist, soll es möglich sein den Roboter ohne direkten Sichtkontakt zu steuern. Hierzu muss eine Kameraübertragung des Roboters auf die Steuercontroller erfolgen.\\

Die Fertigung und Implementierung der Roboter ist Teil einer vorangegangenen Bachelorarbeit, auf die hier nur Oberflächlich eingegangen wird \cite{ALEX}. 

Im ersten Teil der Arbeit werden die verwendeten Technologien näher gebracht. Daraufhin folgt die Diskussion über das Softwarekonzept der Steuerung. Weiter folgt die Implementierung und die tatsächliche Herangehensweise bei der Entwicklung dieser Steuerung und abschließend folgt ein kurzes Fazit und Ausblick auf mögliche Erweiterungen
             



\section{Schnittstelle zum Roboter}
\label{sec:schnittstelle}
Die Schnittstelle zum Roboter definiert auch die Abgrenzung dieser Bachelorarbeit zu der von Alexander Ulbrich. Durch den Pool der möglichen Befehle an den Roboter wird klar, was der Roboter bereits kann und was durch die Steuereinheit erreicht werden muss. 

Da der Roboter eine Byte-Orientierte Kommunikationsweise erfordert, und keine Key-Value Objekte entgegennehmen und versenden kann, mussten die Pakete in Bezug auf die Befehlsreihenfolge für beide Übertragungsrichtungen definiert werden.
Im Folgenden ist die Paketstruktur aufgelistet, die der Roboter vom Server empfängt. \\

\begin{table}[h]
\begin{tabular}{||p{0.15\textwidth}||p{0.22\textwidth}||p{0.525\textwidth}||}
	\hline Bytenummer & Befehl & Beschreibung \\ 
	\hline 0 & Start Byte &  Jeder Befehl beginnt mit 0xFF\\ 
	\hline 1 & Kamera & 0x01 aktiviert die Kamera, 0x00 deaktiviert sie \\ 
	\hline 2 & Schussapparat & 0x01 löst einen Schuss aus, zu schnelle Wiederholungen werden vom Roboter ignoriert \\ 
	\hline 3 & Motorleistung links & Setzt die Leistung des linken Antriebs auf diesen Wert (zwischen 0 und 100 \% in Hex, also 0x00 - 0x64) \\ 
	\hline 4 & Motorleistung rechts & Setzt die Leistung des rechten Antriebs auf diesen Wert (zwischen 0 und 100 \% in Hex, also 0x00 - 0x64) \\ 
	\hline 5 & Checksumme & Dieses Feld ermöglicht die Erkennung von fehlerhaften Paketen. Sowohl dem Server, als auch dem Roboter ist die Berechnung hierfür bekannt. \\
	\hline
\end{tabular}  
\caption{Paketstruktur, die der Roboter annimmt}
\label{tab:serv_to_robot}
\end{table}


Hier wird ersichtlich, dass selbst für die Beschleunigung nach vorne, also gleichmäßiges Beschleunigen der Räder, keine Funktion oder derartiges bereit steht, lediglich die beiden Motoren können unabhängig voneinander angesteuert werden. Dadurch muss jegliche Logik, die es ermöglicht den Roboter in bestimmte Richtungen zu navigieren, von der Steuereinheit bereitgestellt werden. \\

In der anderen Richtung, also Pakete die vom Roboter aus gehen, gibt es lediglich einige Statusnachrichten und die Bilddaten. Auch diese sind in Tabelle \ref{tab:robot_to_serv} aufgeführt.



\begin{table}
	\begin{tabular}{||p{0.15\textwidth}||p{0.22\textwidth}||p{0.525\textwidth}||}
		\hline Bytenummer & Befehl & Beschreibung \\ 
		\hline 0 & Start Byte &  Jeder Befehl beginnt mit 0xFF\\ 
		\hline 1 & Akkustand & Enthält den aktuellen Ladestand des Akkus (zwischen 0 und 100 \% in Hex, also 0x00 - 0x64) \\ 
		\hline 2 & Goal-Flag & Enthält Werte zwischen 0x00 und 0x03. Das entspricht den Situationen: Tor in Sicht, Richtungsweisung in Sicht (links und rechts) und kein Tor in Sicht \\ 
		\hline 3 - 9 & Nicht-Implementiert & An dieser Stelle waren ursprünglich mehrere Stati eingeplant, jedoch aus Zeit- und Aufwandgründen nicht weiter beachtet \\ 
		\hline 10 + 11 & Bildlänge & In diesen beiden Bytes wird die Bildlänge mitgeteilt, die das Paket ab diesem Byte mit sich bringt \\ 
		\hline 12 - ? & Bilddaten & Je nach Bedarf werden noch zusätzliche Bilddaten an das Paket angehängt \\
		\hline
	\end{tabular}  
	\caption{Paketstruktur, die der Roboter versendet}
	\label{tab:robot_to_serv}
\end{table}                         
                                                                                                                                                                                                                                                                                                                                                                                                                                                                                                                                                                                                                                                                                                                                                                                                                                                                                                                                                                                                                                                                                                                                                                                                                                                                                                                                                                                                                                                                                                                                                                                                                                                                                                                                                                                                                                                                                                                                                                                                                                                                                                                                                                                                                                                                                                                                                                                                                                                                                                                                                                                                                                                                                                                                                                                                                                                                                                                                                                                                                                                                                                                                                                                                                                                                                                             
                                                                                                                                                                                                                                                                                                                                                                                                                                                                                                                                                                                                                                                                                                                                                                                                                                                                                                                                                                                                                                                                                                                                                                                                                                                                                                                                                                                                                                                                                                                                                                                                                                                                                                                                                                                                                                                                                                                                                                                                                                                                                                                                                                                                                                                                                                                                                                                                                                                                                                                                                                                                                                                                                                                                                                                                                                                                                                                                                                                                                                                                                                                                                                                                                                                                                                                                                                                                                                                                                                                                                                                                                                                                                                                                                                                                                                                          
                                                                                                                                                                                                                                                                                                                                                                                                                                                                                                                                                                                                                                                                                                                                                                                                                                                                                                                                                                                                                                                                                                                                                                                                                                                                                                                                                                                                                                                                                                                                                                                                                                                                                                                                                                                                                                                                                                                                                                                                                                                                                                                                                                                                                                                                                                                                                                                                                                                                                                                                                                                                                                                                                                                                                                                                                                                                                                                                                                                                                                                                                                                                                                                                                                                                                                                                                                                                                                                                                                                                                                                                                                                                                                                                                                                                                                                                                                                                                                                                                                                                                                                                                                                                                                                                                                                                                                                                                                                                                                                                                                                                                                                                                                                                                                                                                                                                                                