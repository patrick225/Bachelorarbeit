\chapter*{Abstract}
Automation und eingebettete Systeme gehören in der heutigen Zeit bereits zum Standard. In der Industrie sind Roboter schon lange nicht mehr wegzudenken. Sie erledigen sowohl hoch präzise, als auch robuste, kraft-aufwendige Arbeiten wesentlich genauer und schneller als ein Mensch es je können wird. Ein weiteres Einsatzgebiet von Robotern sind für Menschen unzugängliche oder gefährliche Orte, wie zum Beispiel mit Gas gefüllte Höhlen oder sogar Orte außerhalb der Erdatmosphäre (zB. Mars-Rover). Doch ein weiterer, immer stärker anwachsender Anwendungszweig von Robotern ist die Unterhaltungsbranche. Diese Bachelorarbeit beschäftigt sich mit der Ansteuerung und Bedienbarkeit eines eigens entwickelten Fussballroboters. \\
Die Konstruktion der akkubetriebenen Roboter inklusive Softwareschnittstelle ist Thema einer weiteren Bachelorarbeit, auf die hier nur verwiesen wird.

Ziel dieser Bachelorarbeit ist es, ein Fussballspiel zwischen zwei Benutzern zu ermöglichen. Hierbei bilden Smartphones oder andere Computer mit einem Webbrowser die Fernsteuerung für die Roboter. Über diese Fernsteuerung ist es möglich die Roboter zu steuern und mithilfe eines am Roboter befestigten Schussapparats, ein Tor zu erzielen. Am Spielfeldrand sind wie im Fussball üblich Tore aufgestellt. Fällt ein Tor, findet eine Torerkennung statt und der entsprechende Spielstand wird an dem Bediengerät angezeigt. Außerdem findet eine Kameraübertragung von den Robotern zu den Bediengeräten statt, sodass das Spiel auch gespielt werden kann, ohne sich in unmittelbarer Nähe der Roboter zu befinden. Ebenfalls Teil der Arbeit ist es zu gewährleisten, dass das Spiel auch ohne manuelles Eingreifen spielbar bleibt. Dafür ist es notwendig, dass die Roboter bei niedrigem Akkustand selbstständig die Ladestation aufsuchen und sich automatisch laden. 
Damit die Roboter frei und ohne Hindernisse fahren können, findet die Kommunikation über Funk statt. Hierbei wird eine Client-Server Architektur gewählt, die es erlaubt, unabhängig von den Robotern unterschiedliche Bediengeräte zu implementieren. Ein weiterer Vorteil dabei ist, dass auch eine Kommunikation zwischen Server und Roboter stattfinden kann, ohne dass eine Fernbedienung verbunden ist, was für das selbstständige Anfahren der Ladestation wichtig ist. Das Finden der Ladestation funktioniert mittels einer Infrarot-LED, die impulslängenmodulierte Signale sendet.