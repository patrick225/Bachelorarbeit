\chapter{Einführung der Komponenten}

In diesem Kapitel findet eine kurze Einführung in die verwendeten Komponenten statt. Im restlichen Teil der Arbeit wird immer wieder auf diese Rollen zurück gegriffen.

\section{Roboter}
Wie in Kapitel \ref{ch:Grundlagen} eingeführt wurde der Roboter im Rahmen einer weiteren Bachelorarbeit erstellt. Hier soll nur kurz auf die für diese Arbeit wichtigen Aspekte des Roboters eingegangen werden. \\
Der Roboter verfügt über zwei Räder, die über jeweils einen Motor beschleunigt und abgebremst werden können. Außerdem befindet sich vorne ein Schussmechanismus, der mit einem Befehl (vgl. \ref{sec:schnittstelle}) ausgelöst werden kann. Die beiden Motoren der Räder beschleunigen den Roboter auf bis zu $2\frac{m}{s}$. Dadurch wird ersichtlich, dass bei hohen Geschwindigkeiten eine gute Reaktionszeit notwendig ist, damit sich der Roboter noch steuern lässt. Auf der hinteren Seite des Roboters befinden sich Ladekontakte und knapp darüber ein Infrarot-Empfänger. Ein Bild des Roboters befindet sich im Anhang \ref{fig:robot_pic}. Für diese Arbeit werden zwei Roboter benötigt.

\section{Server}
An dieser Stelle wird die Entscheidung aus Kapitel \ref{ch:software-architektur} vorgegriffen um später besser mit den Begrifflichkeiten umgehen zu können. \\
Der Server ist ein Java-Programm, das die zentrale Kommunikationsstelle darstellt. Er trennt die Controller von der Logik und von der direkten Kommunikation mit den Robotern. Dieses Programm wird auf einem RaspberryPi ausgeführt, der sich mit einer festen IP-Adresse im Universitätsnetz befindet. Über die GPIO-Pins des RaspberryPi's sind die Unterkomponenten des Tors angeschlossen.

\section{Tor}
Das Tor ist ein mithilfe von 3D-Druck erstelltes Gehäuse, das die IR-LED, die Ladestation und die Torerkennung vereint. Pro Roboter wird ein Tor aufgestellt, das genau diesem Roboter zugeordnet ist.

\subsection{Ladestation}
Mit der Ladestation sind die Ladekontakte gemeint, die am Tor befestigt sind. Sie befinden sich auf Höhe der Ladekontakte des Roboters, um ein Laden des Roboters zu ermöglichen, nur indem er gegen die Kontakte des Tors fährt.

\subsection{IR-LED}
Die Infrarot-LED (IR-LED) ist die Komponente, die notwendig ist, um dem Roboter zu signalisieren wo sich das Tor und damit die Ladestation befindet. Tatsächlich handelt es sich hierbei um drei Infrarot-LEDs die zu einem Bauteil zusammengefasst wurden. Sie befinden sich am oberen Rand des Tore, auf Höhe des IR-Empfängers des Roboters.

\subsection{Torerkennung}
Die Torerkennung ist der mechanische Schalter, der ein Signal an den Server weiterleitet sobald ein Tor erzielt wurde. 

\section{Steuercontroller}
Unter einem Steuercontroller verstehen wir das Gerät, das der Nutzer verwendet um den Roboter zu steuern. Das kann entweder ein Android-Gerät sein, auf dem die Android-Anwendung ausgeführt wird oder die Webanwendung mithilfe eines Laptops oder PCs. 
