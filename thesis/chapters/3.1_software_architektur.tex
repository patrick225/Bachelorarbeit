\section{Software Architektur}
\label{sec:software-architektur}

Die Software-Architektur des Projektes definiert die Möglichkeiten, die man zur Implementierung der Anforderungen zur Verfügung hat. Deshalb war es wichtig, sich ausreichend Gedanken zu machen, um später im Projekt dann nicht feststellen zu müssen, dass die gewählte Architektur für die Anforderungen ungeeignet ist. Die Hauptanforderung an die Architektur ist selbstverständlich die Bereitstellung eines Kommunikationskanals zwischen Controllern und Robotern. Es muss möglich sein Richtungs- und Geschwindigkeitsänderungen seitens der Anwender an die Roboter mitzuteilen. Der Status des Roboters soll jederzeit ausgewertet und entsprechend darauf reagiert werden können. Hierzu gehören das automatische Anfahren der Ladestation bei geringem Akkustand und die Übertragung der Bilddaten. Außerdem sollen Zustandsänderungen des Spiels, die nicht vom Roboter oder Controller direkt ausgehen erkannt und korrekt verarbeitet werden. 

Ausgehend von diesen Bedingungen, die die Architektur erfüllen muss, kamen folgende Architekturen in Frage:

Eine \textbf{Client-Server-Architektur} bei der ein Server zentral die Kommunikation verwaltet. Hierbei findet keine Kommunikation zwischen den Steuergeräten und den Robotern direkt statt. Die Torerkennung kommuniziert direkt mit dem Server, ohne dass Roboter und Controller zwangsweise etwas davon mitbekommen. Sowohl die Roboter, als auch die Controller nehmen die Rolle eines Clients ein. Dadurch ist es Möglich, dass eine Steuerung des Servers stattfinden kann, ohne dass ein Controller verbunden ist. 