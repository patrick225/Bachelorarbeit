\chapter{Fazit und Ausblick}
\label{ch:fazit}

Zum Ende der Arbeit erfolgt nun ein Fazit, indem auf die in der Einleitung definierten Ziele noch einmal eingegangen und diskutiert werden soll, ob die Vorgehensweise zieldienlich war. Anschließend gibt es einen Ausblick auf mögliche Erweiterungen und Verbesserungen der bestehenden Komponenten.

\section{Fazit}
Das Ziel der Arbeit war es, eine Steuerung für einen eigens entwickelten Roboter zu entwerfen, die es dem Benutzer ermöglicht mit seinem Smartphone oder Laptop den Roboter vor Ort, oder mithilfe der Kameraübertragung zu steuern, damit ein Fussballspiel ermöglicht werden kann. Dabei spielt es eine tragende Rolle, dass die Rechtzeitigkeit im Sinne der Übertragungszeit für Kamerabild und Befehlslatenz sicher gestellt ist. Hierfür war es notwendig eine geeignete Übertragungstechnologie zu finden und die Schnittstelle zum Roboter effizient zu gestalten. Auch die Aufgabe, den Roboter wartungsfrei zu halten was den Ladevorgang angeht, galt es zu bewältigen. \\
Die verzögerungsfreie Bedienbarkeit des Roboters über die Steuercontroller bestätigt die Rechtzeitigkeit in Bezug auf die Steuerung. Dies war zu erwarten, da die maximale Verzögerung von 100ms (vgl. \ref{sec:wahl_frequenz}) zwischen dem Wunsch und dem Eintritt einer Richtungsänderung zu keiner merklichen Beeinträchtigung des Spielflusses führen. Die Kameraübertragung jedoch hat aufgrund der niedrigen Framerate des Roboters gewisse Verzögerungen. Diese Verzögerungen führen bei einer hohen Geschwindigkeit des Roboters zu erschwertem Handling, die jedoch nicht von der Übertragung, sondern von der Roboterhardware zu verantworten sind. \\
Die Aufgabe den Roboter selbstständig an die Ladestation zu führen wurde meiner Meinung nach nur teilweise erfüllt. So ist es zwar möglich die Ladestation zu finden und auch anzufahren, jedoch hängt der Erfolg noch am Treffen der Ladekontakte. Da dies jedoch nur äußere Umstände sind, die von einer geschickteren Ladestation ausgemerzt werden könnten und die Station ja tatsächlich gefunden wird, halte ich diesen Punkt für wenig relevant, da er sich außerhalb des Interessengebiets befindet. \\
Somit lässt sich abschließend sagen, dass es durchaus möglich ist einen eigens entwickelten Fussballroboter ohne für den Nutzer erkennbare Verzögerungen zu steuern. Lediglich die Bildübertragung bringt eine gewisse Unrechtzeitigkeit ins Spiel, die aber vermutlich nicht an der Übertragung von Server zu Controller hängt, sondern an der Übertragungskapazität des Roboters. Wäre diese höher ausgefallen, hätte man sich sicherlich für ein Streamingprotokoll entscheiden müssen.\\
Die Anforderungen für diese Arbeit wurden jedoch zufriedenstellend erfüllt. Es wurde eine Steuerung für ein Roboter Fussballspiel entwickelt, bei dem die Roboter mithilfe von mobilen Endgeräten gesteuert werden.





\section{Ausblick}

Das Thema Steuerung von Robotern bietet generell viel Spielraum wenn es um Erweiterungen geht. Zunächst wird jedoch auf die bereits verwendeten Komponenten eingegangen.\\
Auf die Ladestation in Bezug auf das Design der Kontakte wird im Nachfolgenden nicht weiter eingegangen, da dies den Rahmen des eigentlichen Themas zu sehr übertreten würde. Jedoch wäre für die Torfindung an sich eine Positionsbestimmung des Roboters sehr hilfreich. Der Roboter müsste sich nicht ständig neu orientieren und könnte auf Anhieb den richtigen Weg zur Station finden. Dies ließe sich zum Beispiel mit einer über dem Spielfeld hängenden Kamera realisieren.  \\
Auch die Torerkennung bietet noch Verbesserungspotenzial. So könnte man die Tore mit Torkameras ausstatten, die den Ball erkennen. Das schließt Fehlerkennungen durch Rempler oder Fremdkörper aus. \\

Führt man diesen Gedanken Positionsbestimmung weiter, so wäre es dadurch auch möglich eine zentrale Steuerung seitens des Servers zu entwickeln. Hierbei würde der Server alle Steuerbefehle anhand der aktuellen Position des Roboters errechnen und diese an den Roboter senden, also eine Art Computergegner, um das Spiel auch ohne zweiten menschlichen Gegner spielen können. \\
Die Funktion den Roboter von außerhalb des Universitätsnetzwerks zu steuern ist zwar vorhanden, jedoch benötigt der Server hierfür auch eine öffentliche IP-Adresse. Wird ihm eine solche zugeteilt, wäre die Teilnahme am Spiel von jedem Ort mit Internetzugang möglich.