\documentclass[10pt,a4paper,oneside]{scrartcl}

% \usepackage[ngerman]{babel}   % Deutsches Proposal
 \usepackage[USenglish]{babel} % English proposal

\usepackage[utf8]{inputenc}
\usepackage{hyperref,xcolor,microtype,ifthen}
\setkomafont{disposition}{}
\setkomafont{descriptionlabel}{\bfseries}

\usepackage[textwidth=450pt]{geometry}

% Kommentare abschalten
\newboolean{showhints}
\setboolean{showhints}{true}
% \setboolean{showhints}{false}

\newcommand\hint[2]{
\ifthenelse{\boolean{showhints}}{
\begin{center}
\colorbox{black!10}{
\begin{minipage}{.963\textwidth}
#2\hfill\textbf{#1}
\end{minipage}
}\end{center}}{}
}

\title{Titel der Arbeit}
\subtitle{(Bachelorarbeit / Masterarbeit) Proposal}
\author{Vorname Nachname}

\begin{document}

\maketitle

\section{Motivation}
\label{sec:motivation}

\subsection{\iflanguage{ngerman}
	{Forschungs- und Arbeitsgebiet}
	{Research field}}
\label{sub:field}

\hint{(diese Seite)}{Allgemeines Forschungs-/Arbeitsgebiet (z.B. VANETs, IMD, Simulation) in dem die Arbeit eingeordnet ist einleiten, Charakteristika und Anwendungsgebiete aufzeigen und, wo sinnvoll, einführende Literatur zitieren.}

\dots

\subsection{\iflanguage{ngerman}
	{Bereich der Arbeit}
	{Thesis area}}
\label{sub:topic}

\hint{(0,5 Seiten)}{Spezielles Themengebiet der Arbeit (z.B. Kommunikationsprotokolle, Sicherheit, Privacy, Missbrauchserkennung) in das Forschungsgebiet einordnen, Fragestellungen aufzeigen. Hier sollen auch mögliche Anforderungen und Ziele für das Ergebnis der eigenen Arbeit diskutiert werden (z.B. notwendige Effizienz).}

\dots


\section{\iflanguage{ngerman}
	{Verwandte Forschung und ähnliche Ansätze}
	{State of the Art}}
\label{sec:state_of_the_art}

\hint{(1 Seite)}{Einschlägige Literatur zum speziellen Thema vorstellen. Die Literaturauswahl erfolgt in Absprache mit dem Betreuer und muss nicht vollständig das Thema abdecken. Vorgestellte Literatur sollte Probleme von (oder sogar das fehlen von) existierenden Lösungen für das Thema der Arbeit aufzeigen.}

\dots

\section{\iflanguage{ngerman}
	{Problemstellung}
	{Problem Statement}}
\label{sec:problem_statement}

\subsection{\iflanguage{ngerman}
	{Beitrag der Arbeit}
	{Thesis focus}}
\label{sub:focus}

\hint{(0,75 Seiten)}{Auf Basis der Motivation (\ref{sec:motivation}) und der existierenden Forschung (\ref{sec:state_of_the_art}) soll die Problemstellung der eigenen Arbeit in Prosa beschrieben werden. Dazu sollen die Artefakte, die entstehen sollen (z.B. Konzept, Implementierung, Literaturstudie) sowie die Evaluations-Methodologie beschrieben werden (z.B. formale Analyse, Simulation, Benutzerstudie).}

\dots

\subsection{\iflanguage{ngerman}
	{Fragestellungen}
	{Research questions}}
\label{sub:questions}

\hint{(0,25 Seiten)}{Die hier aufgebrachten Fragen sind je nach Ausrichtung der Arbeit entweder
Forschungsfragen oder Umsetzungsfragen. Arbeiten mit Wissenschaftscharakter zielen eher auf
Forschungsfragen ab. Arbeiten mit Innovationscharakter haben dagegen zum Inhalt wie existierende
Technologien für konkrete Fragestellungen neuartig angewandt werden.  

	In beiden Fällen beschreiben die Fragen die gesamte Problemstellung in einer oder wenigen
Hauptfragen sowie weiteren Teilfragen. Die Fragen sowie der generelle Fokus der Arbeit (Forschungs- 
oder Innovationsgetrieben) werden zusammen mit dem Betreuer erarbeitet. Forschungsfragen sind als 
W-Fragen (Wie, Warum, Was, Weshalb, \dots) formuliert und lassen sich in Typen unterteilen.%
\footnote{\url{http://www.studieren.at/articles/497/1/Unterschiedliche-Typen-von-Forschungsfragen/}}
%
\begin{description}
\item[Erklärung:] Warum ist etwas so wie es ist?
\item[Beschreibung:] Wie sieht der jeweilige Sachverhalt aus?
\item[Prognose:] Wie wird sich etwas entwickeln?
\item[Bewertung:] Wie ist das Beschriebene/Erklärte zu bewerten?
\item[Gestaltung:] Was muss getan werden, um ein Ziel zu erreichen?
\end{description}

Die Fragen sollen so formuliert sein, dass sie im Rahmen der Arbeit beantwortet werden können und die gesamte Problemstellung umfassen.
}

\noindent\emph{Hauptfrage \dots}
%
\begin{enumerate}
	\item Teilfrage
	\item Teilfrage
	\item \dots
\end{enumerate}

\section{\iflanguage{ngerman}
	{Eigener Ansatz}
	{Approach}}
\label{sec:approach}

\hint{(0,5 Seiten)}{Erste Ideen für die eigene Arbeit und erste Ansätze, um die Forschungsfragen zu beantworten. Ideen können Zeiger auf vielversprechende Literatur sein (muss nicht komplett gelesen sein) sowie eine Skizze wie man die Literatur anwenden möchte. Der Ansatz muss nicht konkret sein, sondern ist gedacht als Ausblick welche Ansätze man in den ersten Wochen nach der Anmeldung verfolgt.}

\dots


\section{\iflanguage{ngerman}
	{Planung}
	{Planning}}
\label{sec:planning}

\subsection{\iflanguage{ngerman}
	{Eigene Vorkenntnisse}
	{Own Background}}
\label{sub:background}

\hint{(0,25 Seiten)}{Relevantes Vorwissen zu Methodiken und Hilfsmitteln (z.B. Simulationsumgebung, passende formale Modelle, Algorithmen) aus gehörten Vorlesungen, Seminaren und Nebenjobs. Notwendige Fähigkeiten, die während der Arbeit erlernt werden sollen.}

\dots

\subsection{\iflanguage{ngerman}
	{Benötigte Ressourcen}
	{Required Resources}}
\label{sub:resources}

\hint{(beliebig lang)}{Falls zutreffend, eine Liste von benötigter Hardware oder anderen besonderen Ressourcen.}

\begin{description}
\item[Ressource] Beschreibung \dots
\item[Ressource] Beschreibung \dots
\end{description}

\subsection{\iflanguage{ngerman}
	{Zeitplanung}
	{Work packages}}
\label{sub:wp}

\hint{(0,25 Seiten)}{Grobe Einteilung der Arbeitspakete auf Monatsebene (bei 6-wöchigen Bachelorarbeiten auf Wochenebene). Die Zeitplanung soll konsistent sein mit den Forschungsfragen (\ref{sub:questions}) sowie möglicher Einarbeitungszeit (\ref{sub:background}) und möglichen Wartezeiten auf Ressourcen (\ref{sub:resources}).}

\begin{description}
\item[M1] \dots
\item[M2] \dots
\item[M3] \dots
\item[M4] \dots
\item[M5] \dots
\item[M6] \dots
\end{description}

\subsection{\iflanguage{ngerman}
	{Risiken und Ausweichplan}
	{Contingency plan}}
\label{sub:contingency}

\hint{(0,5 Seiten)}{Mögliche Risiken (z.B. nicht verfügbare Ressourcen, kein effizienter Algorithmus existent) die bereits vor Beginn der Arbeit abgeschätzt werden können sowie Ausweichstrategien.}

\begin{enumerate}
\item Risiko \dots

      Ausweichplan \dots

\item Risiko \dots

      Ausweichplan \dots
\end{enumerate}

\nocite{*}
\bibliographystyle{plain}
\bibliography{../literature/references}

\end{document}